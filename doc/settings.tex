\usepackage[left=3cm,right=3cm,top=3cm,bottom=3cm]{geometry} % Page margins
\usepackage[english, ngerman]{babel} % Better line breaking
\usepackage[utf8]{inputenc} % Utf8 recognition
\usepackage[T1]{fontenc} % Translate from latex code to draw font
\usepackage{lmodern} % Bolder font
\usepackage{graphicx} % Display images
\usepackage{fancyhdr} % Header/footer
\usepackage[pdfborder={0 0 0}]{hyperref} % Links without visible lines
\usepackage[table]{xcolor} % Get the color in the table
\usepackage{pdflscape} % Get the table into landscape mode
\usepackage[lastpage]{zref} % Set a lable on the last page
\usepackage{listings} % Display formatted code
\usepackage{makeidx} % Generates the index
\usepackage{acronym} % Generates the list of abbreviations
\usepackage{multicol} % List of abbreviations with two columns
\usepackage{bibgerm} % BibTex German Style DIN 1505
\usepackage{longtable} % Multi page tables
\usepackage{subfigure} % Multiple gaphics in one figure
\usepackage{setspace} % set line spacing
\usepackage{amsmath} % Improved Math mode
\usepackage{amsfonts} % Improved Math mode
\usepackage{amssymb} % Improved Math mode
\usepackage{mathtools} % Improved Math mode
\usepackage{csquotes}
\usepackage{upquote}
%\usepackage{underscore}

% Tell LaTeX to generate an index
\makeindex

% No indent @ line start
\parindent 0pt

% The bibstyle
% gerplain is for only numbers in alphabetic order
% geralpha is for name+year in alphabetic order
\bibliographystyle{gerplain}

% Header
\pagestyle{fancy}

\fancypagestyle{plain}{}
\fancyhead[RO,RE]{Seite \thepage}
\fancyhead[LO,LE]{\leftmark}

% Set the lastpage counter -1 
% So the statutory declaration is not part of the page counter
\makeatletter
\newcommand{\reallastpage}{
  \the \numexpr \zref@extractdefault{LastPage}{page}{0}-1\relax
}
\makeatother

% Header for starting section
\fancypagestyle{rightmark}{
	\fancyhead[LO,LE]{\rightmark}
}

% Empty footer
\cfoot{}

% Brake long url in cite
\def\UrlBreaks{\do\/\do-}

% Suppress clubs (Schusterjungen) 
\clubpenalty = 10000

% Suppress widows (Hurenkinder)
\widowpenalty = 10000

% Suppress widows in front of a formular
\displaywidowpenalty = 10000

% Macro for centering extreme wide tables/figures
\makeatletter
\newcommand*{\Centerfloat}{%
  \parindent \z@
  \leftskip \z@ \@plus 1fil \@minus \textwidth
  \rightskip\leftskip
  \parfillskip \z@skip}
\makeatother

% Change the text from the list of listings
\addto\captionsngerman{\renewcommand{\bibname}{Literatur- \& Quellenverzeichnis}}

%------------------------------------------------------------------------------
% Macro for two abstracts on one page:
%------------------------------------------------------------------------------

\newenvironment{Abstract}{
  \vspace*{\fill}
  \begin{center}%
    \bfseries\abstractname
  \end{center}}%
  {\vfill}

%------------------------------------------------------------------------------
% Macro for reusing some text:
%------------------------------------------------------------------------------

% Use to define some text and then re use the very same text
% \ref{sth:bla}
% \textlabel{Something AAA}{sth:bla}
\makeatletter
\newcommand*{\Textlabel}[2]{%
  \edef\@currentlabel{#1}% Set target label
  \phantomsection% Correct hyper reference link
  #1\label{#2}% Print and store label
}
\makeatother

%------------------------------------------------------------------------------
% Macros for the list of abbreviations:
%------------------------------------------------------------------------------

% To wirte the text only once
\newcommand*{\ListOfAbbreviations}{Abkürzungsverzeichnis}

% use reversed form
\makeatletter
\renewcommand*{\@acf}[1]{%
\ifAC@footnote
\acsfont{\AC@acs{#1}}%
\footnote{\AC@placelabel{#1}\hskip\z@\AC@acl{#1}{} }%
\else
\acsfont{% Orig:\acffont
\AC@placelabel{#1}\hskip\z@\AC@acs{#1}%Orig: \AC@acl{#1}
\nolinebreak[3] %
\acfsfont{(\acffont{\AC@acl{#1}})}%Orig: (\acsfont{\AC@acs{#1}})
}%
\fi
\ifAC@starred\else\AC@logged{#1}\fi
}
\makeatother

%------------------------------------------------------------------------------
% Macros for the the Index:
%------------------------------------------------------------------------------

% The thickness of the line between the columns of the index and the list of 
% Abbreviations; 0.4 pt is the LaTeX standart
\newcommand*{\LineThickness}{0.4 pt}

% Display twoculums with  vertical seperator line
\makeatletter
\renewenvironment{theindex}
  {\if@twocolumn
      \@restonecolfalse
   \else
      \@restonecoltrue
   \fi
   \setlength{\columnseprule}{\LineThickness} % Thikness of the columnseprule
   \setlength{\columnsep}{35 pt}
   \begin{multicols}{2}[\chapter*{\indexname}] % Amount of Columns
   \markboth{\MakeUppercase\indexname}%
            {\MakeUppercase\indexname}%
   \thispagestyle{plain}
   \setlength{\parindent}{0pt}
   \setlength{\parskip}{0pt plus 0.3pt}
   \relax
   \let\item\@idxitem}%
  {\end{multicols}\if@restonecol\onecolumn\else\clearpage\fi}
\makeatother

%------------------------------------------------------------------------------
% Macros for table colors and longtables:
%------------------------------------------------------------------------------

% Table gray row
\newcommand*{\Grayrow}{\rowcolor{gray!35}}

% Table red cell
\newcommand*{\Redcell}{\cellcolor{red!30}}

% Table green cell
\newcommand*{\Greencell}{\cellcolor{green!30}}

% Generates a small empty row in a longtable
\newcommand*{\EmptyRow}{\multicolumn{0}{l}{} \\[-9pt]}

% Needs to be @ the end of the longtable, 
% generates a caption with correct spacing
% @par1: the text of the caption
\newcommand*{\CaptionLongtable}[1]
	{\multicolumn{0}{l}{} \\[-3pt]\caption{#1}}
	
% Needs to be @ the end of the longtable, 
% generates a caption with correct spacing
% @par1: the text of the caption
% @par2: the text of the caption in the list of tables
\newcommand*{\CaptionLongtableS}[2]
	{\multicolumn{0}{l}{} \\[-3pt]\caption[#2]{#1}}

%------------------------------------------------------------------------------
% Macros for quotes:
%------------------------------------------------------------------------------

% A direct quote
% @par1: The quoted text
% @par2: The source where the text is from
% @par3: The page where the text is from
\newcommand*{\QuoteDirect}[3]{\QuoteM{\emph{#1}} \cite[#3]{#2}}

% A direct quote without page
% @par1: The quoted text
% @par2: The source where the text is from
\newcommand*{\QuoteDirectNoPage}[2]{\QuoteM{\emph{#1}} \cite{#2}}

% A indirect quote
% @par1: The source where the text is from
% @par2: The page where the text is from
\newcommand*{\QuoteIndirect}[2]{(vgl. \cite[#2]{#1})}

% A indirect quote without page
% @par1: The source where the text is from
\newcommand*{\QuoteIndirectNoPage}[1]{(vgl. \cite{#1})}

% A text with quotation marks
% @par1: The text you want to quote
% »text«
\newcommand*{\QuoteM}[1]{\frqq #1\flqq}

% A text with single quotation marks
% @par1: The text you want to quote
% ›text‹
\newcommand*{\QuoteMs}[1]{\frq #1\flq}

% To adjust some words to the flow
% @par1: the adjusted words
% [text]
\newcommand*{\AdjustWords}[1]{{\normalfont[#1]}}

% Displays a reference to the given object
% @par1: the lable of the thing you want to see
% (Siehe auch Abbildung 1.1 »Ein Bild« auf Seite 4)
\newcommand*{\SeeS}[1]
{(siehe \autoref{#1} \QuoteM{\nameref{#1}})}

% Displays a reference to the given object
% @par1: the lable of the thing you want to see
% (Siehe auch Abbildung 1.1 »Ein Bild« auf Seite 4)
\newcommand*{\SeeB}[1]
{(Siehe auch \autoref{#1} \QuoteM{\nameref{#1}} auf \autopageref{#1})}

% Displays a reference to an equation
% @par1: the lable of the equation you want to see
% (Siehe auch Gleichung 1.1 in »Dummy Section« auf Seite 5)
\newcommand*{\SeeEq}[1]
{(siehe auch \autoref{#1} in \QuoteM{\nameref{#1}} auf \autopageref{#1})}

% The symbol for a elision
% Is used for more than missings one word or a sentence
% [...]
\newcommand*{\Elision}{{\normalfont[\dots]}}

% The symbol for a small elision
% Is used for only one missing word
% [..]
\newcommand*{\ElisionSmall}{{\normalfont[..]}}

% This is used if a book is cited at whole 
% passim means something like continuous
% text (vgl. [Aut99, passim]).
\newcommand*{\passim}{passim}

% To show the audience that there is something
% To display a wrong/importen part but not corrected in the quote
% text error [sic!] text
\newcommand*{\SIC}{{\normalfont[sic!]}}

% The text for a note from the author
% text, Anm. d. Autors
\newcommand*{\NoteFromAuthor}{{\normalfont\unskip , Anm. d. Autors}}

%------------------------------------------------------------------------------
% Listings:
%------------------------------------------------------------------------------

% get chapter spacing right for \lstlistoflistings
\let\Chapter\chapter
\def\chapter{\addtocontents{lol}{\protect\addvspace{10pt}}\Chapter}

% Change the text from a listings caption
\renewcommand*{\lstlistingname}{Listing}

% Change the text from the \autoref
\def\lstlistingautorefname{\lstlistingname}

% Change the text from the list of listings
\renewcommand*{\lstlistlistingname}{Codeverzeichnis}

% C++ code environment
% @par1: The caption
% @par2: The label
% Used as:
%   \begin{c++}{caption}{label}
%      c++ code
%   \end{c++}
% Use empty brackets for code without caption and/or lable like:
%   \begin{c++}{}{} 
%      c++ code
%   \end{c++}
\lstnewenvironment{java}[2]{
	\lstset{ % General command to set parameter(s)
		language=java,
		% The language of the code
		basicstyle=\small \ttfamily,
		% The size of the fonts that are used for the code
		breaklines=true,
		% Sets automatic line breaking
		captionpos=b,
		% Sets the caption-position to bottom
		showstringspaces=false,
		% Underline spaces within strings only
		showspaces=false,
		% Show spaces everywhere adding particular underscores;
		% it overrides 'showstringspaces'
		keepspaces=true,
		% Keeps spaces in text, useful for keeping indentation
		% of code (possibly needs columns=flexible)
		numbers=left,
		% Where to put the line-numbers; 
		% possible values are (none, left, right)
		showtabs=false,
		% Show tabs within strings adding particular underscores
		keywordstyle=\bfseries \color{blue},
		% Keyword style
		rulesepcolor=\color{gray},
		% The color of the shadow of the box
		identifierstyle=\ttfamily,
		% The style for non-keywords
		commentstyle=\bfseries \color{gray},
		% Comment style
		stringstyle=\ttfamily \color{red!50!brown},
		% String literal style
		numberstyle=\tiny,
		% The style that is used for the line-numbers
		tabsize=2,
		% Sets default tabsize to 2 spaces
		frame=shadowbox,
		% Adds a frame around the code use single for no shadow
		rulecolor=\color{black},
		% If not set, the frame-color may be changed
		% on line-breaks within not-black text
		moredelim=[is][\underbar]{__}{__},
		% To create a underlind text to highlight something: __text__
		caption={#1},
		% The caption of the code example will be 
		% shown in the lstlistoflistings
		label={#2}
		% The lable used to make a ref to the code
	}
}{}

% bash code environment
% @par1: The caption
% @par2: The label
% Used as:
%   \begin{bash}{caption}{label}
%      bash code
%   \end{bash}
% Use empty brackets for code without caption and/or lable like:
%   \begin{bash}{}{} 
%      bash code
%   \end{bash}
\lstnewenvironment{bash}[2]{
	\lstset{literate=%
	  {Ö}{{\"O}}1
	  {Ä}{{\"A}}1
	  {Ü}{{\"U}}1
	  {ß}{{\ss}}2
	  {ü}{{\"u}}1
	  {ä}{{\"a}}1
	  {ö}{{\"o}}1
	}
	\lstset{ % General command to set parameter(s)
		language=bash,
		% The language of the code
		basicstyle=\small \ttfamily,
		% The size of the fonts that are used for the code
		breaklines=true,
		% Sets automatic line breaking
		captionpos=b,
		% Sets the caption-position to bottom
		showstringspaces=false,
		% Underline spaces within strings only
		showspaces=false,
		% Show spaces everywhere adding particular underscores;
		% it overrides 'showstringspaces'
		keepspaces=true,
		% Keeps spaces in text, useful for keeping indentation
		% of code (possibly needs columns=flexible)
		numbers=left,
		% Where to put the line-numbers; 
		% possible values are (none, left, right)
		showtabs=false,
		% Show tabs within strings adding particular underscores
		keywordstyle=\bfseries \color{blue},
		% Keyword style
		rulesepcolor=\color{gray},
		% The color of the shadow of the box
		identifierstyle=\ttfamily,
		% The style for non-keywords
		commentstyle=\bfseries \color{gray},
		% Comment style
		stringstyle=\ttfamily \color{red!50!brown},
		% String literal style
		numberstyle=\tiny,
		% The style that is used for the line-numbers
		tabsize=2,
		% Sets default tabsize to 2 spaces
		frame=shadowbox,
		% Adds a frame around the code use single for no shadow
		rulecolor=\color{black},
		% If not set, the frame-color may be changed
		% on line-breaks within not-black text
		moredelim=[is][\underbar]{__}{__},
		% To create a underlind text to highlight something: __text__
		caption={#1},
		% The caption of the code example will be 
		% shown in the lstlistoflistings
		label={#2},
		% The lable used to make a ref to the code
		deletekeywords={complete, wait},
		% Delete bad keywords
		morekeywords={mkdir, wget, cut}
		% Add missing keywords
	}
}{}

% python code environment
% @par1: The caption
% @par2: The label
% Used as:
%   \begin{python}{caption}{label}
%      python code
%   \end{bash}
% Use empty brackets for code without caption and/or lable like:
%   \begin{python}{}{} 
%      python code
%   \end{bash}
\lstnewenvironment{python}[2]{
	\lstset{ % General command to set parameter(s)
		language=python,
		% The language of the code
		basicstyle=\small \ttfamily,
		% The size of the fonts that are used for the code
		breaklines=true,
		% Sets automatic line breaking
		captionpos=b,
		% Sets the caption-position to bottom
		showstringspaces=false,
		% Underline spaces within strings only
		showspaces=false,
		% Show spaces everywhere adding particular underscores;
		% it overrides 'showstringspaces'
		keepspaces=true,
		% Keeps spaces in text, useful for keeping indentation
		% of code (possibly needs columns=flexible)
		numbers=left,
		% Where to put the line-numbers; 
		% possible values are (none, left, right)
		showtabs=false,
		% Show tabs within strings adding particular underscores
		keywordstyle=\bfseries \color{blue},
		% Keyword style
		rulesepcolor=\color{gray},
		% The color of the shadow of the box
		identifierstyle=\ttfamily,
		% The style for non-keywords
		commentstyle=\bfseries \color{gray},
		% Comment style
		stringstyle=\ttfamily \color{red!50!brown},
		% String literal style
		numberstyle=\tiny,
		% The style that is used for the line-numbers
		tabsize=2,
		% Sets default tabsize to 2 spaces
		frame=shadowbox,
		% Adds a frame around the code use single for no shadow
		rulecolor=\color{black},
		% If not set, the frame-color may be changed
		% on line-breaks within not-black text
		moredelim=[is][\underbar]{__}{__},
		% To create a underlind text to highlight something: __text__
		caption={#1},
		% The caption of the code example will be 
		% shown in the lstlistoflistings
		label={#2},
		% The lable used to make a ref to the code
		deletekeywords={},
		% Delete bad keywords
		morekeywords={True, False}
		% Add missing keywords
	}
}{}

\lstnewenvironment{html}[2]{
	\lstset{ % General command to set parameter(s)
		language=html,
		% The language of the code
		basicstyle=\small \ttfamily,
		% The size of the fonts that are used for the code
		breaklines=true,
		% Sets automatic line breaking
		captionpos=b,
		% Sets the caption-position to bottom
		showstringspaces=false,
		% Underline spaces within strings only
		showspaces=false,
		% Show spaces everywhere adding particular underscores;
		% it overrides 'showstringspaces'
		keepspaces=true,
		% Keeps spaces in text, useful for keeping indentation
		% of code (possibly needs columns=flexible)
		numbers=left,
		% Where to put the line-numbers; 
		% possible values are (none, left, right)
		showtabs=false,
		% Show tabs within strings adding particular underscores
		keywordstyle=\bfseries \color{blue},
		% Keyword style
		rulesepcolor=\color{gray},
		% The color of the shadow of the box
		identifierstyle=\ttfamily,
		% The style for non-keywords
		commentstyle=\bfseries \color{gray},
		% Comment style
		stringstyle=\ttfamily \color{red!50!brown},
		% String literal style
		numberstyle=\tiny,
		% The style that is used for the line-numbers
		tabsize=2,
		% Sets default tabsize to 2 spaces
		frame=shadowbox,
		% Adds a frame around the code use single for no shadow
		rulecolor=\color{black},
		% If not set, the frame-color may be changed
		% on line-breaks within not-black text
		moredelim=[is][\underbar]{__}{__},
		% To create a underlind text to highlight something: __text__
		caption={#1},
		% The caption of the code example will be 
		% shown in the lstlistoflistings
		label={#2},
		% The lable used to make a ref to the code
		deletekeywords={},
		% Delete bad keywords
		morekeywords={True, False},
		% Add missing keywords
		escapeinside={(*}{*)}
	}
}{}
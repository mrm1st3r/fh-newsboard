\subsection{Technische Basis}
Umgesetzt wird das Newsboard in Java, sowohl wegen persönlicher Präferenzen der Autoren,
sowie der Tatsache, dass hauptsächlich Java als Programmiersprache
an der FH Bielefeld gelehrt wird und so die Weiterentwicklung des Newsboards
positiv beeinflusst. Darüber hinaus gibt es für Java eine große Zahl an ausgereiften
und gut dokumentierten Bibliotheken, sowie verlässliche Frameworks zum Build-Management.

Als Basis für das Newsboard wird das Spring-Framework im Zusammenspiel mit Maven
als Build-Management-Tool verwendet. Diese Kombination ist in der Praxis bewährt,
ist flexibel für verschiedenste Anwendungsfälle, sowie ohne große Einarbeitungszeit
gut zu benutzen.

\subsection{Datenmodell}

\subsection{Schnittstelle}
Zur Implementierung der REST-Endpoints der Schnittstelle werden die in Spring enthaltenen
Controller-Features genutzt. Um damit einen Endpoint zu bedienen, reicht es aus,
eine Methode innerhalb einer Controller-Klasse mit \texttt{RequestMapping} zu annotieren.
Listing \ref{lst:restExample} zeigt Beispielhaft, wie ein solcher Endpoint unter Angabe der
Annotations-Parameter implementiert werden kann.

Der Rückgabewert der Methode wird so direkt als Antwort an den Client zurückgeliefert.
Darüber hinaus können mit Hilfe der Dependency-Injection von Spring weitere Parameter definiert werden, die z.B. die HTTP-Anfrage, oder die Antwort wie im Beispiel.
\vspace{1em}

\begin{java}{Implementierung eines REST-Endpoints mit Spring}{lst:restExample}
@RestController
public class RestApiController {
	[...]
	@RequestMapping(path = "/document", method = RequestMethod.GET, produces = MediaType.APPLICATION_XML_VALUE)
	public String listDocuments(HttpServletResponse response) {
		[...]
		return documents;
	}
	[...]
}
\end{java}

Über die REST-Endpoints hinaus ist bei der Schnittstelle noch die Verbeitung der XML-Daten,
sowie das Schema zur Validierung interessant.
Da die zu erwartende Datenmengen im Vorhinein nicht abschätzbar sind,
sollte das Lesen und vor allem das Schreiben der XML-Daten möglichst wenig Speicher
verbrauchen, um eventuelle Engpässe zu vermeiden. Beim Lesen ist eine Validierung
gegen das Schema unbedingt notwendig, da nicht davon ausgegangen werden kann,
dass die ankommenden Daten in jedem Fall valide sind.

Aufgrund der Speicheranforderungen ist ein herkömmlicher DOM-Parser in diesem Projekt
nicht geeignet. Gelesen werden die XML-Daten stattdessen mit einem SAX-Parser.


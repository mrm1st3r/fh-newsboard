\subsection{Architektur}
Der erste richtungweisende Punkt für des Konzepts ist die Architektur des Newsboards.
Da eine Webseite zum Einsehen der bewerteten Nachrichten eine der Voraussetzungen ist,
bietet sich eine klassische Client-Server-Architektur an.
Dabei treten neben den Webbrowsern des Benutzer auch die Crawler-
und Bewertungsmodule als Clients auf.

Um die Komplexität des Newsboads möglichst gering zu halten, sollte die Kommunikation
mit den Modulen ebenfalls über HTTP stattfinden. Das bringt den Vorteil mit sich,
dass es für viele Programmiersprachen bereits gute Client-Bibliotheken gibt
und sich so keine unnötigen Einschränkungen ergeben.

\subsection{Datenmodell}

\subsection{Schnittstelle}
Die Hauptziele bei der Konzeption der Schnittstelle waren zum einen eine möglichst
einfache Verwendung, sowie eine verlässliche Validierung der Übertragenen Daten.
Da die Schnittstelle auf HTTP aufbaut, bietet es sich an, REST zu verwenden.

Als Datenformat für die Schnittstelle bieten sich sowohl JSON, als auch XML an.
Da XML im Gegensatz zu JSON allerdings eine verlässliche Möglichkeit zur Validierung
und Dokumentation durch ein Schema besitzt, ist es in diesem Fall die bessere Wahl.
Der geringfügig höhere Speicherverbrauch, sowie die derzeit geringe Popularität
von XML sollen hier keine Rolle spielen.

Bei der Konzeption der verwendeten REST-Ressourcen, sowie deren akzeptierten Methoden
bestehen ebenfalls zwei mögliche Szenarien. Das erste ist die Abbildung sämtlicher
Datentypen des Datenmodells und das Erlauben aller CRUD-Operationen.
Das zweite Szenario ist es, ausschließlich einige wenige Ressourcen zur Verfügung
zu stellen, welche die geplanten Anwendungsfälle abbilden.

Letzteres verlangt komplexere Daten, die von der Schnittstelle verarbeitet werden müssen,
andernfalls läge diese Komplexität jeweils in den einzelnen Clients und müsste
mehrfach konzeptioniert und umgesetzt werden. Das führt letztendlich zu mehr
unnützer Arbeit und gleichzeitig zur mehr potentiellen Fehlerquellen.
Da für diese Schnittstelle potentiell regelmäßig neue Clients entwickelt werden sollen,
ist die zweite Variante als deutlich zweckmäßiger, sowie allgemein eleganter anzusehen.

\subsection{Zusammenfassung}
Durch die Verwendung von Spring, Java und MySQL als Grundlage konnte eine modulare, flexible
und performante Software entwickelt werden. Über die REST-Schnittstelle kann das System
mit unterschiedlichsten Systemen interagieren. 

Ein umfangreiches Testen hat sichergestellt, dass die Funktionen des Newsboards darüber hinaus
robust und einsatzfähig sind.

Die zur Satzerkennung eingesetzte Bibliothek openNLP ist zwar einfach zu
verwenden, die Präzision kann aber noch verbessert werden. Zum jetzigen Zeitpunkt gibt es 
beispielsweise noch Probleme bei der Erkennung einzelner Punkte in einem Satz, beispielsweise
bei einer Abkürzung.

Leider konnte bis zum Projektabschluss keine Volltextsuche implementiert werden. Lediglich ein
Prototyp, basierend auf Apache Lucene wurde implementiert. Für die Suche wurde ein Index
im RAM angelegt, welcher sich nicht für die Verwendung in Umgebungen mit großen Datenmengen
eignet.

\subsection{Ausblick} 
Im jetzigen Stand bietet das Newsboard viele Ansatzpunkte zur Entwicklung neuer 
Funktionalitäten. Neben der Erweiterung der REST-Schnittstelle mit zusätzlichen Optionen ist 
auch eine überarbeitete und umfangreichere Oberfläche denkbar, die in späteren 
Studentenprojekten entwickelt werden können.

Auch eine Verwendung von weiteren Datenbanksystemen ist möglich. Mit den fortschreitenden
Entwicklungen im Bereich der NoSQL-Datenbanken ist es womöglich sinnvoll , dass, in absehbarer
Zeit, eine Alternative zum relationalen Modell eingesetzt wird. 

Die Erkennung von Sätzen könnte ebenfalls in zukünftigen Projekten verbessert werden. Neben
der Verbesserung des eingesetzten Modells könnte auch über eine komplette Eigenentwicklung
nachgedacht werden.

Darüber hinaus ist die Entwicklung einer Volltextsuche in späteren Projekten sinnvoll. Sollte
das Newsboard tatsächlich über einen längeren Zeitraum am Stück eingesetzt werden, können
die Datenmengen schnell ansteigen. Eine performante Volltextsuche in Kombination mit einer
umfangreichen Oberfläche würde einen großen Mehrwert für die Verwendung des Newsboards
bedeuten.

Zurückblickend hat sich die Entwicklung des Newsboards dennoch als erfolgreich gezeigt.
Dank der verwendeten und ausgereiften Technologien wie Spring, Java und MySQL ist das
Newsboard eine robuste und performante Anwendung, die in der Lage ist, ihren Anforderungen
gerecht zu werden. Einem Einsatz steht zum jetzigen Zeitpunkt nichts im Wege.
Abschließend soll nun das Projektergebnis zusammengefasst und ein Ausblick auf mögliche
Weiterentwicklungen gegeben werden.

\subsection{Zusammenfassung}
Durch die Verwendung von Spring, Java und MySQL als Grundlage konnte eine modulare
und flexible Software entwickelt werden. Über die REST-Schnittstelle kann das System
mit unterschiedlichsten Systemen interagieren. Ein umfangreiches Testen hat sichergestellt,
dass die Funktionen des Newsboards darüber hinaus robust und einsatzfähig sind.

Die zur Satzerkennung eingesetzte Bibliothek openNLP ist zwar einfach zu
verwenden, die Präzision kann aber noch verbessert werden. Zum jetzigen Zeitpunkt gibt es 
noch Probleme bei der Erkennung einzelner Punkte in einem Satz, beispielsweise bei einer
Abkürzung.

Leider konnte bis zum Projektabschluss keine Volltextsuche implementiert werden. Lediglich
ein Prototyp, basierend auf Apache Lucene wurde implementiert. Für die Suche wurde ein Index
im RAM angelegt, welcher sich nicht für die Verwendung in Umgebungen mit großen Datenmengen
eignet.

Die entwickelte Oberfläche ist zwar ausreichend, um die Funktion des Newsboards
zu demonstrieren, kann allerdings noch nicht als vollständig bezeichnet werden.
So fehlt zum Beispiel noch eine Administrationsfunktion, um neue Crawler oder Classifier
hinzuzufügen. Aufgrund des sauberen Aufbaus der Anwendung sollte dies jedoch einfach
nachträglich durch Erweitern oder Ersetzen der Oberfläche möglich sein.

Außerdem können dank der REST-Schnittstelle externe Anwendungen, wie Single-Page-Apps,
oder mobile Anwendungen auf der Newsboard zugreifen und alternative Oberflächen bereitstellen.

\subsection{Ausblick} 
Im jetzigen Stand bietet das Newsboard viele Ansatzpunkte zur Entwicklung neuer 
Funktionalitäten. Neben der Erweiterung der REST-Schnittstelle mit zusätzlichen Optionen ist 
auch eine überarbeitete und umfangreichere Oberfläche denkbar, die in späteren 
Studentenprojekten entwickelt werden können.

Auch eine Verwendung von weiteren Datenbanksystemen ist möglich. Mit den fortschreitenden
Entwicklungen im Bereich der NoSQL-Datenbanken ist es womöglich sinnvoll, dass in
absehbarer Zeit eine Alternative zum relationalen Modell eingesetzt wird.

Die Erkennung von Sätzen könnte ebenfalls in zukünftigen Projekten verbessert werden. Neben
Optimierungen des eingesetzten Modells ist auch eine komplette Eigenentwicklung zur
Satzextraktion denkbar.

Darüber hinaus ist die Entwicklung einer Volltextsuche in späteren Projekten sinnvoll. 
Sollte das Newsboard tatsächlich über einen längeren Zeitraum eingesetzt werden, 
können die Datenmengen schnell ansteigen. Eine performante Volltextsuche in Kombination mit 
einer umfangreichen Oberfläche würde einen großen Mehrwert für die Verwendung des Newsboards
bedeuten.
